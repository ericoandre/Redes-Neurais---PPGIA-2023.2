\begin{resumo} 
A cultura da fava tem recebido pouca atenção por parte dos órgãos de pesquisa e extensão, o que tem resultado em limitações do conhecimento das características agronômicas da cultura. Isso tem afetado a precisão em classificá-las. Tal classificação é de grande importância porque a identificação correta de plantas permite boa resposta da cultura em termos de produtividade e comportamento em diferentes condições ambientais. Neste contexto de informações limitadas sobre as características apresentamos uma solução que aplica o poder da visão computacional à agronomia, que visa melhorar a produtividade, reduzir desperdícios e auxiliar na tomada de decisões e na seleção de cultura que mais se adequá a uma região em particular. As técnicas de visão computacional são um conjunto de métodos utilizados para interpretar imagens, extraindo padrões e características. Visando contribuir com esse cenário de desenvolvimento tecnológico do setor do agronômico, este trabalho compara algumas das abordagens de classificação supervisionada para identificação de forma automática de espécies de favas. O escopo deste trabalho consiste em classificar imagens de mudas geradas por produtores rurais em duas categorias de favas: orelha de vó e cearense. A partir das comparações realizadas entre métodos de classificadores que utilizam redes convolucionais como extratores de características com diferentes classsificadores como máquina de vetores de suporte (SVM), arvore de descição e mlp, para ao final apresentarmos a melhor métodologia para automatizar a classificação das imagens de favas.
\end{resumo}