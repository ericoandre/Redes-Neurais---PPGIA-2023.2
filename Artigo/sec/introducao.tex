\section{Introdução} 

%A agricultura faz parte do setor primário da economia sendo fornecedor de alimento e de matéria-prima. Com a modernização e profissionalização, o setor deu um salto de importância na economia Brasileira. Atualmente é responsável por quase um quarto do Produto Interno Bruto (PIB)~\cite{FieldView}, e praticamente metade das exportações do país. 

A fava (\textit{Phaseolus lunatus} L.) também conhecida como feijão-lima ou feijão-fava é uma leguminosa cultivada em quase todas as regiões do mundo, sendo que no Brasil possui ampla distribuição em todo o território, principalmente no Nordeste, é uma das quatro espécies do gênero \textit{Phaseolus} exploradas comercialmente. A espécie foi domesticada na América do Sul ou Central, ou em ambas, e é subtropical \cite{zimmermann1988origem}. É uma das principais leguminosas cultivadas na região tropical e apresenta potencial para fornecer proteína vegetal à população, diminuindo a dependência quase exclusiva dos feijões do grupo carioca \cite{vieira1992cultura}. 

%A cultura da fava apresenta-se como fonte de renda e alimento, sendo um recurso natural disponível e resistente aos períodos de seca \cite{revistacultivar:2015}. Permitindo que comunidades rurais, desestimuladas pela exploração agropecuária limitada principalmente pela falta de água, possam recuperar o desejo de permanência no campo, possibilitando melhoria na qualidade de vida, segurança alimentar e nutricional para a agricultura familiar apesar das adversidades climáticas.

%Destacada a importância do feijão-fava à agricultura familiar, buscamos classificar de forma precocemente as mudas, evitando assim o desperdício de tempo e recurso com variedades que tenham uma baixa produtividade. Pois foi verificado que as mudas de feijão-fava são praticamente idênticas nas etapas iniciais tornando difícil mesmo para especialistas classificá-las.

A humanidade vem busca por meios que maximize o processo agrícola, evitando disperdi-cios e aumentando a produtividade agrícola, com a classificação das mudas de forma rápida e correta e possível escolher o melhor manuseio e um maior controle da produção.

A importância de se identificar as espécies de fava se dá porque cada variedade tem características que se diferem uma das outras como o tamanho de vagens e quantidade de grãos que podem variar para mais ou menos por safra, já a variedade orelha de vó tem o maior comprimento de suas vagens e maior peso de sementes que variedade fava-cearense. A escolha dessas duas espécies foi para resolver um problema do curso de agronomia da Universidade Federal Rural de Pernambuco (UFRPE). Que era classificar de forma rápida e precisa as variedades de Fava Orelha de Vó e Fava Cearense. 

Espera-se por meio desta pesquisa elaborar um métodos baseados em redes convolucionais com extrator de características na classificação de variedades de fava. Este trabalho foca-se na construção de um agente classificador utilizando redes convolucionais como extrator de características em diferentes abordagem de classificadores para distinguir espécies de feijão-lima ou feijão-fava, dentre elas a fava cearense e a orelha de vó auxiliando o agricultor na escolha do manejo adequado a espécie classificada.

%Espera-se por meio desta pesquisa contribuir com o processo de classificação de determinadas variedades de fava como orelha de vó e fava Cearense, através da imagens podem ser obtidas a partir de dispositivos portáteis, trazendo economia com essa automação, neste contexto, a análise de imagens tem função valiosa ao transformar os dados coletados no campo em informações úteis na tomada de decisão dos agricultores com a construção de um classificador automático.

 \label{sec:introducao}