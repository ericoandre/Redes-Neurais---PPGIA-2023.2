\section{Conclusão}

Foi possível perceber que todos os métodos testados erraram poucas instância, indicando uma boa precisão nas classificações no experimento como mostra na tabela~\ref{tabela:metricas}. De modo geral, os resultados com SVM foram superiores aos com o classificador baseado em Softmax e Árvore de decisão. Onde a máquina de vetores de suporte (SVM) obteve uma precisão de 0.98 com 6 erros apenas 6 instâncias foi onde se obteve os melhores resultados seguido pela Árvore de decisão que apresentou uma precisão de 0.96.

\begin{table}[H]
\centering
\caption{Contem as métricas precisão e Sensibilidade}
\label{tabela:metricas}
\def\arraystretch{1.2}
\begin{tabular}{@{}lrrrrrrr@{}}
\toprule
{\textbf{Algoritimos}} & {\textbf{Precisão}} & {\textbf{Cobertura }} & {\textbf{AUC-ROC}}\\
\midrule
CNN & 0.946429 & 0.968037 & 0.966  \\ 
CNN SVM & 0.981308 & 0.990566 & 0.988  \\ 
CNN TREE & 0.964286 & 0.986301 & 0.979 \\
VGG16 MLP & 0.958762 & 0.963730 & 0.974 \\
% VGG16 & 0.963636 & 0.968037 \\ 
\bottomrule
\end{tabular}
\end{table}

Os sistemas de visão computacional têm sido cada vez mais utilizados na agricultura facilitando a tomada de decisão dos agricultores. O presente trabalho apresentou um método para identificação de espécies e classificação de mudas de feijão-fava. Neste trabalho, quatro modelos foram comparados com a finalidade de determinar qual seria o melhor classificador. Como visto no capítulo anterior, a junção de SVM com as camadas de convolução na etapa de extração de características se mostrou eficaz, com apenas 6 erros obtendo uma precisão de 0.98 e cobertura de 0.99.

Como trabalhos futuros, pretendemos explorar mais variedades como boca-de-moça, branquinha, mororó, olho-de-ovelha, olho-de-peixe, raio-de-sol, rajada vermelha, rajada preta. Obter recursos de hardware mais robustos, para estender a seleção de modelos.

\label{sec:conclusao}